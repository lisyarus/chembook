\documentclass{book}

% for dots in table of contents
\usepackage{tocloft}
\renewcommand{\cftchapdotsep}{\cftdotsep}

% for clickable table of contents
\usepackage{hyperref}
\hypersetup{
	colorlinks = true,
	linkcolor = blue,
	citecolor = magenta,
}

% for appendices in table of contents
\usepackage[titletoc]{appendix}

% for references section in table of contents
\usepackage[nottoc]{tocbibind}

% for proper name of references section
\renewcommand{\bibname}{References}

\usepackage[a4paper,margin=1in]{geometry}

\newcommand{\molecule}[1]{\textrm{#1}}

% for nice monospaced font using \texttt
\usepackage{courier}

\usepackage{enumitem}

\usepackage{titlesec}

\titlespacing{\section}{0pt}{\parskip}{\parskip}

\setlist{nosep}
\setlength{\parskip}{1em}

\title{Quantum Chemistry In One Weekend}
\author{Nikita Lisitsa \\ \texttt{lisyarus@gmail.com}}
\date{circa January 2020}

\begin{document}

\maketitle

\tableofcontents

\chapter{Introduction} \label{chap:intro}

\section{What it is all about}
It was already a month of me trying to come up with a perfect introduction for this book when I decided I better do something more practical. Let's get straight to the point: what is this book about? Or, rather, what is it \textit{not} about? This is \textit{not} a

\begin{itemize}
\item Chemistry textbook
\item Quantum mechanics textbook
\item Linear algebra textbook
\item Python tutorial
\end{itemize}

But what it \textit{is}, then? Well, a sturdy mix of all the above, aimed at getting physically significant results as soon as possible. We are going to write some Python code using \href{https://numpy.org}{\texttt{numpy}} and \href{https://www.scipy.org}{\texttt{scipy}} libraries (and ocasionally \href{https://matplotlib.org}{\texttt{matplotlib}}, if you want some fancy plots \& images) and obtain energies and orbital densities for some pretty tiny chemical systems (that is, systems composed of nuclei \& electrons).

The reader is expected to have \textit{at least} a bit of knowledge on physics/chemistry, basic mathematics and Python programming language. For getting into the physics and maths involved, I recommend either reading the first few chapters of \cite{ref:atkins} or not skipping the \nameref{chap:intro} chapter. For learning Python, I recommend googling some tutorials/courses/etc, for there are way too many of them to recommend anything in particular.

The following sections of the \nameref{chap:intro}  chapter briefly discuss some essential theoretical background. I'd especially recommend reading the \nameref{sec:var} section so that you understand what all those matrices even mean and what those solvers are actually solving. The actual coding starts at the \nameref{chap:one} chapter.

Beware that I am by no means a scientist, a quantum chemist or even a physicist. I am barely an amateur possessing some unconventional hobbies. The methods that we are going to make use of are \textbf{not} the methods used by real quantum chemical software and are applicable to tiny systems only (with, say, 4 or 5 electrons, while systems of practical interest can contain \textit{thousands} of them).

The name of this book was inspired by Peter Shirley's \href{https://raytracing.github.io}{Ray Tracing in One Weekend series}, which is an amazing source for learning about modern photorealistic rendering.

Finally, feel free to \href{mailto:lisyarus@gmail.com}{email me} if you have some questions, I would love to be of help.

\section{A tale of energies}
\section{A classical failure}
\section{Variational method} \label{sec:var}
\section{Practical considerations}
\section{Harmonic oscillator}
\section{Hydrogen atom \(\molecule{H}\)}
\section{Helium cation \(\molecule{He}^+\)} 
\section{Lithium 2+ cation \(\molecule{Li}^{2+}\)}
\section{Dihydrogen cation \(\molecule{H}_2^+\)}
\section{Trihydrogen 2+ cation \(\molecule{H}_3^{2+}\)}

\begin{thebibliography}{9}
\bibitem{ref:atkins}
	Peter Atkins, Ronald Friedman,
	\textit{Molecular Quantum Mechanics}
\bibitem{ref:lieb}
	Elliot Lieb,
	\textit{The Stability of Matter: From Atoms to Stars}
\bibitem{ref:hall}
	Brian Hall,
	\textit{Quantum Theory for Mathematicians}
\bibitem{ref:kato-th}
	Tosio Kato,
	\textit{On the eigenfunctions of many-particle systems in quantum mechanics}
\bibitem{ref:kato-ham}
	Tosio Kato,
	\textit{Fundamental properties of Hamiltonian operators of Schrödinger type}
\bibitem{ref:fulton-harris}
	William Fulton, Joe Harris,
	\textit{Representation Theory: A First Course}
\bibitem{ref:edmonds}
	A. R. Edmonds,
	\textit{Angular Momentum in Quantum Mechanics}
\bibitem{ref:tmpchem}
	TMP Chem,
	\url{www.youtube.com/channel/UC3dZQdfv67X49cZkoXWYSwQ}
\bibitem{ref:psi4numpy}
	Psi4NumPy,
	\url{www.github.com/psi4/psi4numpy}
\end{thebibliography}

\begin{appendices}
\chapter{Hermite polynomials} \label{apx:hermite}
\chapter{Derivation of GTO integrals} \label{apx:integrals}
\end{appendices}

\end{document}
